% **
% Anexos
% *

\anexo{Exemplo de implementação do método de Koblitz com a biblioteca ecutils (Script em Bash)} \label{anexo:0eca4a2b-8a60-41b3-8d25-2f8056b51dfd}
\begin{verbatim}
#!/bin/bash

M_1="Matematica"

echo -n "M_1 = {"

for ((i=0; i<${#M_1}; i++)); do
    echo -n "${M_1:i:1}" 
    if [ $i -lt $((${#M_1}-1)) ]; then
        echo -n ", " 
    fi
done

echo "}" 

echo -n "A_1 = {"

for ((i=0; i<${#M_1}; i++)); do
    echo -n "$(printf "%d" "'${M_1:i:1}'")"
    if [ $i -lt $((${#M_1}-1)) ]; then
        echo -n ", "
    fi
done

echo "}"

P=$(./ecutils -eck -eck-ec-get "secp192r1" \
    -eck-encode \
    -eck-encoding-type "ascii" \
    -eck-encode-message "$M_1")

J=$(echo "$P" | cut -d' ' -f3)

J_decimal=$(echo "ibase=16; $J" | bc)

echo "J = $J_decimal"

Px=$(echo "$P" | cut -d' ' -f1)
Py=$(echo "$P" | cut -d' ' -f2)

Px_decimal=$(echo "ibase=16; $Px" | bc)
Py_decimal=$(echo "ibase=16; $Py" | bc)

echo "P = ($Px_decimal, $Py_decimal)"

M_2=$(./ecutils -eck -eck-ec-get "secp192r1" \
    -eck-decode \
    -eck-encoding-type "ascii" \
    -eck-decode-px "$Px" \
    -eck-decode-py "$Py" \
    -eck-decode-j "$J"
)

echo -n "A_2 = {"

for ((i=0; i<${#M_2}; i++)); do
    echo -n "$(printf "%d" "'${M_2:i:1}'")"
    if [ $i -lt $((${#M_2}-1)) ]; then
        echo -n ", "
    fi
done

echo "}"

echo -n "M_2 = {"

for ((i=0; i<${#M_2}; i++)); do
    echo -n "${M_2:i:1}" 
    if [ $i -lt $((${#M_2}-1)) ]; then
        echo -n ", " 
    fi
done

echo "}" 
    
\end{verbatim}

\anexo{Exemplo de implementação do protocolo de Diffie-Hellman com a biblioteca ecutils (Script em Bash)} \label{anexo:f9430204-a55e-4726-a4ca-167391db1ca3}
\begin{verbatim}
#!/bin/bash

Gx=4
Gy=5

echo "G = ($Gx, $Gy)"

d_A=9

echo "d_A = $d_A"

H_A=$(./ecutils -ec -ec-define \
    -ec-define-p "B" \
    -ec-define-a "1" \
    -ec-define-b "1" \
    -ec-trapdoor \
    -ec-trapdoor-k "$d_A" \
    -ec-trapdoor-gx "$Gx" \
    -ec-trapdoor-gy "$Gy")

H_Ax=$(echo "$H_A" | cut -d' ' -f1)
H_Ay=$(echo "$H_A" | cut -d' ' -f2)

echo "H_A = ($H_Ax, $H_Ay)"

d_B=5

echo "d_B = $d_B"
    
H_B=$(./ecutils -ec -ec-define \
    -ec-define-p "B" \
    -ec-define-a "1" \
    -ec-define-b "1" \
    -ec-trapdoor \
    -ec-trapdoor-k "$d_B" \
    -ec-trapdoor-gx "$Gx" \
    -ec-trapdoor-gy "$Gy")

H_Bx=$(echo "$H_B" | cut -d' ' -f1)
H_By=$(echo "$H_B" | cut -d' ' -f2)

echo "H_B = ($H_Bx, $H_By)"

S_A=$(./ecutils -ec -ec-define \
    -ec-define-p "B" \
    -ec-define-a "1" \
    -ec-define-b "1" \
    -ec-trapdoor \
    -ec-trapdoor-k "$d_A" \
    -ec-trapdoor-gx "$H_Bx" \
    -ec-trapdoor-gy "$H_By")

S_Ax=$(echo "$S_A" | cut -d' ' -f1)
S_Ay=$(echo "$S_A" | cut -d' ' -f2)

echo "S_A = ($S_Ax, $S_Ay) : S_A = d_A * H_B"

S_B=$(./ecutils -ec -ec-define \
    -ec-define-p "B" \
    -ec-define-a "1" \
    -ec-define-b "1" \
    -ec-trapdoor \
    -ec-trapdoor-k "$d_B" \
    -ec-trapdoor-gx "$H_Ax" \
    -ec-trapdoor-gy "$H_Ay")

S_Bx=$(echo "$S_B" | cut -d' ' -f1)
S_By=$(echo "$S_B" | cut -d' ' -f2)

echo "S_B = ($S_Bx, $S_By) : S_B = d_B * H_A"

\end{verbatim}

\anexo{Exemplo de implementação do protocolo de Massey-Omura com a biblioteca ecutils (Script em Bash)} \label{anexo:db563818-c103-45d8-8c07-544c5ad502b5}
\begin{verbatim}
#!/bin/bash

M_Ax=4
M_Ay=5

echo "M_A = ($M_Ax, $M_Ay)"

d_A=9
d_A_i=B

echo "d_A = $d_A"
    
H_A=$(./ecutils -ec -ec-define \
    -ec-define-p "B" \
    -ec-define-a "1" \
    -ec-define-b "1" \
    -ec-trapdoor \
    -ec-trapdoor-k "$d_A" \
    -ec-trapdoor-gx "$M_Ax" \
    -ec-trapdoor-gy "$M_Ay")

H_Ax=$(echo "$H_A" | cut -d' ' -f1)
H_Ay=$(echo "$H_A" | cut -d' ' -f2)

echo "H_A = ($H_Ax, $H_Ay)"

d_B=5
d_B_i=3

echo "d_B = $d_B"

H_B=$(./ecutils -ec -ec-define \
    -ec-define-p "B" \
    -ec-define-a "1" \
    -ec-define-b "1" \
    -ec-trapdoor \
    -ec-trapdoor-k "$d_B" \
    -ec-trapdoor-gx "$H_Ax" \
    -ec-trapdoor-gy "$H_Ay")

H_Bx=$(echo "$H_B" | cut -d' ' -f1)
H_By=$(echo "$H_B" | cut -d' ' -f2)

echo "H_B = ($H_Bx, $H_By)"

echo "d_A_i = $d_A_i"
S_A=$(./ecutils -ec -ec-define \
    -ec-define-p "B" \
    -ec-define-a "1" \
    -ec-define-b "1" \
    -ec-trapdoor \
    -ec-trapdoor-k "$d_A_i" \
    -ec-trapdoor-gx "$H_Bx" \
    -ec-trapdoor-gy "$H_By")

S_Ax=$(echo "$S_A" | cut -d' ' -f1)
S_Ay=$(echo "$S_A" | cut -d' ' -f2)

echo "S_A = ($S_Ax, $S_Ay)"

echo "d_B_i = $d_B_i"
M_A_2=$(./ecutils -ec -ec-define \
    -ec-define-p "B" \
    -ec-define-a "1" \
    -ec-define-b "1" \
    -ec-trapdoor \
    -ec-trapdoor-k "$d_B_i" \
    -ec-trapdoor-gx "$S_Ax" \
    -ec-trapdoor-gy "$S_Ay")

M_A_2x=$(echo "$M_A_2" | cut -d' ' -f1)
M_A_2y=$(echo "$M_A_2" | cut -d' ' -f2)

echo "M_A = ($M_A_2x, $M_A_2y)"

\end{verbatim}

\anexo{Exemplo de implementação do protocolo de ECDSA com a biblioteca ecutils (Script em Bash)} \label{anexo:c054fabc-cd0a-4724-b7e3-63d39dc9fa55}
\begin{verbatim}
#!/bin/bash

# Parâmetros da curva elíptica
ec_p_hex="5"
ec_p_decimal=$(echo "ibase=16; $ec_p_hex" | bc)

ec_a_hex="1D5D"
ec_a_decimal=$(echo "ibase=16; $ec_a_hex" | bc)

ec_b_hex="3CB"
ec_b_decimal=$(echo "ibase=16; $ec_b_hex" | bc)

ec_gx_hex="0"
ec_gx_decimal=$(echo "ibase=16; $ec_gx_hex" | bc)

ec_gy_hex="1"
ec_gy_decimal=$(echo "ibase=16; $ec_gy_hex" | bc)

ec_n_hex="7"
ec_n_decimal=$(echo "ibase=16; $ec_n_hex" | bc)

ec_h_hex="1"
ec_h_decimal=$(echo "ibase=16; $ec_h_hex" | bc)

# Exibir informações da curva elíptica
echo "Parâmetros da curva elíptica:"
echo "p = $ec_p_decimal"
echo "a = $ec_a_decimal"
echo "b = $ec_b_decimal"
echo "n = $ec_n_decimal"
echo "h = $ec_h_decimal"
echo "G = ($ec_gx_decimal, $ec_gy_decimal)"

# Chave privada e mensagem
d_hex="3"
d_decimal=$(echo "ibase=16; $d_hex" | bc)

M_hex="3C5B"
M_decimal=$(echo "ibase=16; $M_hex" | bc)

# Exibir chave privada e mensagem
echo "Chave privada (d) = $d_decimal"
echo "Mensagem (M) = $M_decimal"

# Gerar chave pública (Q)
Q=$(./ecutils -ecdsa -ecdsa-ec-define \
    -ecdsa-ec-define-p "$ec_p_hex" \
    -ecdsa-ec-define-a "$ec_a_hex" \
    -ecdsa-ec-define-b "$ec_b_hex" \
    -ecdsa-ec-define-gx "$ec_gx_hex" \
    -ecdsa-ec-define-gy "$ec_gy_hex" \
    -ecdsa-ec-define-n "$ec_n_hex" \
    -ecdsa-ec-define-h "$ec_h_hex" \
    -ecdsa-private-key "$d_hex" \
    -ecdsa-get-public-key)

Qx_hex=$(echo "$Q" | cut -d' ' -f1)
Qx_decimal=$(echo "ibase=16; $Qx_hex" | bc)

Qy_hex=$(echo "$Q" | cut -d' ' -f2)
Qy_decimal=$(echo "ibase=16; $Qy_hex" | bc)

# Exibir chave pública (Q)
echo "Chave pública (Q) = ($Qx_decimal, $Qy_decimal)"

# Assinar a mensagem (M)
U=$(./ecutils -ecdsa -ecdsa-ec-define \
    -ecdsa-ec-define-p "$ec_p_hex" \
    -ecdsa-ec-define-a "$ec_a_hex" \
    -ecdsa-ec-define-b "$ec_b_hex" \
    -ecdsa-ec-define-gx "$ec_gx_hex" \
    -ecdsa-ec-define-gy "$ec_gy_hex" \
    -ecdsa-ec-define-n "$ec_n_hex" \
    -ecdsa-ec-define-h "$ec_h_hex" \
    -ecdsa-private-key "$d_hex" \
    -ecdsa-signature \
    -ecdsa-signature-message "$M_hex")

R_hex=$(echo "$U" | cut -d' ' -f1)
R_decimal=$(echo "ibase=16; $R_hex" | bc)

S_hex=$(echo "$U" | cut -d' ' -f2)
S_decimal=$(echo "ibase=16; $S_hex" | bc)

# Exibir assinatura
echo "Assinatura:"
echo "R = $R_decimal"
echo "S = $S_decimal"

# Verificar a assinatura
V=$(./ecutils -ecdsa -ecdsa-ec-define \
    -ecdsa-ec-define-p "$ec_p_hex" \
    -ecdsa-ec-define-a "$ec_a_hex" \
    -ecdsa-ec-define-b "$ec_b_hex" \
    -ecdsa-ec-define-gx "$ec_gx_hex" \
    -ecdsa-ec-define-gy "$ec_gy_hex" \
    -ecdsa-ec-define-n "$ec_n_hex" \
    -ecdsa-ec-define-h "$ec_h_hex" \
    -ecdsa-verify-signature \
    -ecdsa-verify-signature-public-key-px "$Qx_hex" \
    -ecdsa-verify-signature-public-key-py "$Qy_hex" \
    -ecdsa-verify-signature-r "$R_hex" \
    -ecdsa-verify-signature-s "$S_hex" \
    -ecdsa-verify-signature-signed-message "$M_hex")

# Verificar o resultado da verificação e exibir apropriado
if [ "$V" == "1" ]; then
    echo "A assinatura é válida."
else
    echo "A assinatura é inválida."
fi
\end{verbatim}

\anexo{Exemplo de implementação da soma e multiplicação de pontos em uma Curva Elíptica sobre um campo finito (Script em Python)} \label{anexo:9a31fc56-bcc9-4d1a-82bb-8fad31be217e}
\begin{verbatim}
def elliptic_curve_addition(P, Q):
    """
    Realiza a adição de dois pontos em uma curva elíptica.

    Args:
        P (tuple): As coordenadas do ponto P na forma (x, y).
        Q (tuple): As coordenadas do ponto Q na forma (x, y).

    Returns:
        tuple: As coordenadas do ponto resultante da adição,
         na forma (x, y).
            Se a adição resultar no ponto no infinito,
            retorna (None, None).
    """
    (x_1, y_1) = P
    (x_2, y_2) = Q

    if P == (None, None):
        return Q

    if Q == (None, None):
        return P

    if P == Q:
        n = (3 * x_1**2 + a) % p
        d = (2 * y_1) % p
        try:
            inv = pow(d, -1, p)
        except:
            return (
                None,
                None,
            )  # Ponto no infinito
        s = (n * inv) % p
        x_3 = (s**2 - x_1 - x_1) % p
        y_3 = (s * (x_1 - x_3) - y_1) % p
        return (x_3, y_3)
    else:
        x_2, y_2 = Q
        n = (y_2 - y_1) % p
        d = (x_2 - x_1) % p
        try:
            inv = pow(d, -1, p)
        except:
            return (
                None,
                None,
            )  # Ponto no infinito
        s = (n * inv) % p
        x_3 = (s**2 - x_1 - x_2) % p
        y_3 = (s * (x_1 - x_3) - y_1) % p
        return (x_3, y_3)


def elliptic_curve_multiplication(k, G):
    """
    Realiza a multiplicação de um ponto por um escalar em
    uma curva elíptica.

    Args:
        k (int): O escalar pelo qual o ponto será multiplicado.
        G (tuple): As coordenadas do ponto G na forma (x, y),
         que é a base da curva.

    Returns:
        tuple: As coordenadas do ponto resultante da
        multiplicação, na forma (x, y).
    """
    if k == 0 or k >= n:
        raise ValueError("k não está no intervalo 0 < k < n")

    R = None

    num_bits = k.bit_length()

    for i in range(num_bits - 1, -1, -1):
        if R is None:
            R = G
            continue

        if R == (None, None):
            R = G

        R = elliptic_curve_addition(R, R)

        if (k >> i) & 1:
            if R == (None, None):
                R = G
            else:
                R = elliptic_curve_addition(R, G)

    return R
\end{verbatim}

\anexo{Exemplo de implementação do método de Koblitz (Script em Python)} \label{anexo:efaa994a-3caf-4704-bf42-c98f2418dab0}
\begin{verbatim}
def koblitz_encode(message: str) -> tuple:
    """
    Codifica uma mensagem utilizando o algoritmo de
    Koblitz.

    Args:
        message: A mensagem a ser codificada.

    Returns:
        Uma tupla contendo o ponto codificado (x, y) e
        o valor de j.

    """
    # Definindo os parâmetros do algoritmo
    alphabet_size = (
        2**8
    )  # Tamanho do alfabeto, neste caso ASCII de 8 bits
    message_length = len(
        message
    )  # Tamanho da mensagem de entrada
    message_decimal = sum(
        ord(message[k]) * alphabet_size**k
        for k in range(message_length)
    )  # Convertendo a mensagem em um número decimal
    # Encontrando um ponto válido na curva de Koblitz
    d = 100
    for j in range(1, d - 1):
        x = (
            d * message_decimal + j
        ) % p  # Para cada valor de j, calcula-se um
        # valor x
        s = (
            x**3 + a * x + b
        ) % p  # Calcula-se o valor de s utilizando a
        # fórmula da curva elíptica
        if s == pow(
            s, (p + 1) // 2, p
        ):  # Verifica se s é um quadrado perfeito
            # (respeita a equação do ponto na curva)
            y = pow(
                s, (p + 1) // 4, p
            )  # Encontra o valor de y utilizando a fórmula
            # da curva elíptica
            if (y**2) % p == (
                x**3 + a * x + b
            ) % p:  # Verifica se o ponto (x, y)
                # está na curva
                break

    # Retorna o ponto codificado e o valor de j
    return (x, y), j


def koblitz_decode(point, j) -> str:
    """
    Decodifica um ponto codificado utilizando o algoritmo
     de Koblitz.

    Args:
        point: O ponto codificado (x, y).
        j: O valor de j utilizado na codificação.

    Returns:
        A mensagem decodificada.

    """
    # Decodifica o ponto (x, y) e o valor j para a mensagem
    # original
    d = 100
    message_decimal = (
        point[0] - j
    ) // d  # Calcula o valor decimal da mensagem utilizando
    # o valor de x, o valor de j e o parâmetro d
    alphabet_size = (
        2**8
    )  # Tamanho do alfabeto, neste caso ASCII de 8 bits
    decoded_message = []
    while (
        message_decimal != 0
    ):  # Converte o número decimal em uma sequência de
        # caracteres
        decoded_message.append(
            chr(message_decimal % alphabet_size)
        )
        message_decimal //= alphabet_size

    # Retorna a mensagem decodificada como uma string
    return "".join(decoded_message)


def elliptic_curve_addition(P, Q):
    """
    Realiza a adição de dois pontos em uma curva elíptica.

    Args:
        P (tuple): As coordenadas do ponto P na forma (x, y).
        Q (tuple): As coordenadas do ponto Q na forma (x, y).

    Returns:
        tuple: As coordenadas do ponto resultante da adição,
         na forma (x, y).
            Se a adição resultar no ponto no infinito,
            retorna (None, None).
    """
    (x_1, y_1) = P
    (x_2, y_2) = Q

    if P == (None, None):
        return Q

    if Q == (None, None):
        return P

    if P == Q:
        n = (3 * x_1**2 + a) % p
        d = (2 * y_1) % p
        try:
            inv = pow(d, -1, p)
        except:
            return (
                None,
                None,
            )  # Ponto no infinito
        s = (n * inv) % p
        x_3 = (s**2 - x_1 - x_1) % p
        y_3 = (s * (x_1 - x_3) - y_1) % p
        return (x_3, y_3)
    else:
        x_2, y_2 = Q
        n = (y_2 - y_1) % p
        d = (x_2 - x_1) % p
        try:
            inv = pow(d, -1, p)
        except:
            return (
                None,
                None,
            )  # Ponto no infinito
        s = (n * inv) % p
        x_3 = (s**2 - x_1 - x_2) % p
        y_3 = (s * (x_1 - x_3) - y_1) % p
        return (x_3, y_3)
\end{verbatim}

\anexo{Exemplo de implementação do protocolo de ECDSA (Script em Python)} \label{anexo:cb388af7-af2f-47f9-b21f-1d0d0a95c9e9}
\begin{verbatim}
from random import randint


def ecdsa_generate_signature(private_key, message_hash):
    """
    Gera uma assinatura ECDSA para uma determinada chave
    privada e hash de mensagem.

    Args:
    private_key (int): A chave privada.
    message_hash (int): O hash da mensagem a ser assinada.

    Returns:
    tuple: A assinatura ECDSA (r, s).

    """
    (r, s) = (0, 0)
    while r == 0 or s == 0:
        k = randint(1, n)
        (x, _) = elliptic_curve_multiplication(k, G)
        r = x % n
        s = (
            (message_hash + r * private_key) * pow(k, -1, n)
        ) % n
    return r, s


def ecdsa_verify_signature(public_key, message_hash, r, s):
    """
    Verifica a validade de uma assinatura ECDSA.

    Args:
    public_key (tuple): O ponto de chave pública (x, y).
    message_hash (int): O hash da mensagem assinada.
    r (int): O primeiro componente da assinatura.
    s (int): O segundo componente da assinatura.

    Returns:
    bool: True se a assinatura é válida, False caso contrário.

    """
    w = pow(s, -1, n)
    u_1 = (message_hash * w) % n
    u_2 = (r * w) % n
    X = elliptic_curve_addition(
        elliptic_curve_multiplication(u_1, G),
        elliptic_curve_multiplication(u_2, public_key),
    )
    v = X[0] % n
    return v == r    
\end{verbatim}

