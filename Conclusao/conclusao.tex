% **
% Conclusão.
% *
\section{CONSIDERAÇÕES FINAIS}
Ao longo deste trabalho, buscou-se desvendar a abordagem da criptografia,
lançando uma luz sobre a complexidade subjacente aos processos rotineiros de
comunicação segura. A criptografia, em suas variadas formas, permeia nosso
cotidiano de tal maneira que sua influência passa quase completamente
despercebida, apesar de sua importância crítica.

Os exemplos apresentados nas seções dedicadas à implementação prática da
criptografia de Curvas Elípticas em um sistema de Chat seguro têm caráter
ilustrativo. Eles procuraram demonstrar a aplicação conjunta dos protocolos de
criptografia estudados, embora otimizações adicionais e a análise de
vulnerabilidades de segurança não tenham sido nosso principal escopo neste
trabalho.

Além disso, neste estudo, levou-se a cabo a elaboração de uma biblioteca em
Python, chamada \textit{Elliptic Curve Utils (ecutils)}. Esta biblioteca
implementa os algoritmos estudados e permite a execução das operações
matemáticas básicas nas curvas elípticas. É um instrumento valioso para quem
está ingressando no estudo da criptografia e pode ser facilmente acessada e
instalada através do repositório
GitHub\footnote{\url{https://github.com/isakruas/ecutils}} ou do \textit{Python
    Package Index}\footnote{\url{https://pypi.org/project/ecutils}} (PyPI).

É fundamental ressaltar que este estudo não esgotou todos os aspectos teóricos acerca
da criptografia e suas aplicações. Nosso objetivo principal foi fornecer uma introdução
abrangente ao tema e um trampolim para mais pesquisas em profundidade. As referências
listadas oferecem excelentes pontos de partida para aqueles que desejam se aprofundar
ainda mais nas camadas de complexidade.

A importância da criptografia em nossas vidas não pode ser subestimada. Ela é a
força que nos permite interagir com sistemas informatizados com um nível
razoável de segurança e confiança. Diante disso, este estudo pode ser
considerado um primeiro passo em direção a uma maior compreensão e apreciação
desta disciplina crucial. Ainda há muito terreno a ser coberto, principalmente
na fronteira emergente da criptografia e da computação quântica.

Finalmente, respaldado pelos tópicos abordados neste trabalho, é empolgante
imaginar as possibilidades futuras para a criptografia. Diante do contínuo e
vertiginoso avanço das tecnologias digitais e computacionais, acreditamos que a
criptografia apenas crescerá em importância e relevância. Ao continuar
pesquisando e construindo sobre os fundamentos estabelecidos aqui, podemos
contribuir para a evolução de um mundo digital mais seguro e confiável.
