\section{INTRODUÇÃO}

Esta pesquisa foi motivada pela necessidade de entender a estrutura e
funcionamento da criptografia de ponta a ponta no contexto digital. Questões
como: \textit{"Como funciona a criptografia de ponta a ponta?"}, e
\textit{"Como podemos assegurar a autenticidade do remetente e a integridade da
	mensagem?"} serviram como base para o desenvolvimento deste estudo.

A criptografia é uma ferramenta indispensável na garantia da privacidade e
segurança das comunicações no mundo digital. No entanto, o desenvolvimento e
implementação de sistemas de criptografia eficientes e seguros ainda consiste
em um desafio para muitos desenvolvedores e organizações. Sistemas de
criptografia, como os baseados em curvas elípticas, possuem uma complexidade
matemática intrínseca que pode tornar seu entendimento e implementação um
processo complexo.

O objetivo desta pesquisa é desmistificar o funcionamento da criptografia
baseada em curvas elípticas e fornecer uma implementação passo a passo deste
sistema. Este estudo é relevante pois pode servir de guia para desenvolvedores
e pesquisadores que necessitam compreender a matemática por trás deste tipo de
criptografia e saber como aplicá-la em situações práticas.

Além disso, considerando o papel primordial da criptografia nos dias atuais
para a proteção de dados e garantia da privacidade no ambiente digital, este
trabalho também tem como propósito contribuir para a literatura na área de
segurança cibernética e criptografia. A pesquisa vai além de apenas apresentar
fundamentos teóricos, propondo também uma implementação prática que possa ser
facilmente reproduzida por outros profissionais da área.

A criptografia de curva elíptica tem se destacado como uma alternativa
eficiente em relação a outras técnicas, devido ao seu desempenho
consideravelmente superior com chaves de tamanho reduzido. Dessa forma, um
estudo aprofundado sobre o uso de curvas elípticas na criptografia traz uma
contribuição significativa tanto para a comunidade acadêmica quanto para a
indústria.

Finalmente, esta investigação parte de um questionamento genuíno e uma busca
por um entendimento técnico e prático dos conceitos que governam as aplicações
de segurança digital. A busca por respostas a estas perguntas possibilita uma
compreensão mais aprofundada dos sistemas criptográficos e permite uma análise
mais crítica do cenário atual da segurança de dados.