% **
% Metodologia.
% *
\section{METODOLOGIA}

A metodologia desta pesquisa foi estruturada em quatro etapas distintas:

Na primeira etapa, realizou-se uma revisão bibliográfica abrangente, na qual se
analisaram e interpretaram trabalhos científicos, monografias, teses,
dissertações e normas de instituições renomadas, como o National Institute of
Standards and Technology (NIST) dos EUA. O foco principal dessa revisão foi a
investigação da criptografia de ponta a ponta, curvas elípticas, corpos finitos
e os fundamentos matemáticos dos algoritmos de criptografia. Essa revisão
bibliográfica proporcionou uma sólida base teórica para o desenvolvimento das
etapas subsequentes do projeto.

Na segunda etapa, exploraram-se as propriedades dos algoritmos de criptografia
de ponta a ponta, com um enfoque particular naqueles baseados em curvas
elípticas. A análise concentrou-se em questões como a codificação e
decodificação de mensagens, o estabelecimento de chaves seguras, a troca de
informações e a assinatura digital. Essa análise teórica permitiu uma
compreensão mais aprofundada das características e dos desafios associados à
criptografia.

A terceira etapa envolveu o desenvolvimento de um \textit{software} simulador,
utilizando a linguagem de programação Python como plataforma de implementação.
A escolha do Python deveu-se à sua robustez e acessibilidade em temas de
segurança da informação e criptografia. Essa etapa transformou a teoria em
prática, permitindo a criação de um ambiente de simulação para os algoritmos de
criptografia estudados.

A quarta etapa contemplou a análise dos resultados gerados pelo
\textit{software} implementado. Realizaram-se testes de eficácia, eficiência e
confiabilidade dos algoritmos e protocolos utilizados. Durante essa fase,
observações e limitações identificadas nos testes foram cuidadosamente
documentadas e analisadas. Essa avaliação experimental forneceu insights
valiosos sobre o desempenho dos algoritmos em um ambiente prático.
