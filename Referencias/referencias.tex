% **
% Referências.
% *
ABREU, Jacqueline de Souza. \textbf{Passado, Presente e Futuro da Criptografia
    Forte}: desenvolvimento tecnológico e regulação. \textit{Revista Brasileira de
    Políticas Públicas}, Brasília, ano 2017, v. 7, n. 3, p. 24-42, 9 dez. 2017. DOI
\url{http://doi.org/10.5102/rbpp.v7i3.4869}. Disponível em:
\url{https://www.publicacoes.uniceub.br/RBPP/article/view/4869}. Acesso em: 15
out. 2023.

ANDRIA, Sally; GONDIM, Rodrigo; SALOMÃO, Rodrigo. \textbf{Introdução à
    Criptografia com Curvas Elípticas}. Rio de Janeiro: IMPA, 2019. 139 p.
Disponível em:
\url{https://impa.br/wp-content/uploads/2022/03/32CBM09_eBook.pdf}. Acesso em:
12 out. 2023.

BRADY, Renee; DAVIS, Naleceia; TRACY, Anna. \textbf{Encrypting with Elliptic
    Curve Cryptography}. \textit{Florida AM University. Naleceia Davis. Spelman
    College. Anna Tracy. University of the South}, 2010. Disponível em:
\url{https://www.msri.org/msri_ups/531/schedules/4117}. Acesso em: 21 jan.
2022.

ERTAUL, Levent; LU, Weimin. ECC Based Threshold Cryptography for Secure Data
Forwarding and Secure Key Exchange in MANET (I). In: \textbf{Networking 2005.
    Networking Technologies, Services, and Protocols; Performance of Computer and
    Communication Networks; Mobile and Wireless Communications Systems}. Berlin,
Heidelberg: Springer Berlin Heidelberg, 2005. p. 102-113. ISBN:
978-3-540-32017-3. DOI \url{http://doi.org/10.1007/11422778_9}. Disponível em:
\url{https://link.springer.com/chapter/10.1007/11422778_9}. Acesso em: 15 out.
2023.

FILHO, José Eustáquio Ribeiro Vieira; AZEREDO, Paula Prestes. Tecnologia,
criptografia e matemática: da troca de mensagens ao suporte em transações
econômicas. \textbf{Desenvolvimento Socioeconômico em Debate}, [S. l.], v. 2,
n. 2, p. 22–31, 2017. DOI: 10.18616/rdsd.v2i2.3225. Disponível em:
\url{https://periodicos.unesc.net/ojs/index.php/RDSD/article/view/3225}. Acesso
em: 12 out. 2023.

GONZAGA, Raoni do Nascimento; PESCO, Sinésio. \textbf{Bitcoin: uma introdução à
    matemática das transações}. Rio de Janeiro: [s.n.], 2021. 65 p. Dissertação
(Mestrado em Matemática) - Pontifícia Universidade Católica do Rio de Janeiro.
Disponível em: \url{https://www.maxwell.vrac.puc-rio.br/54118/54118.PDF}.
Acesso em: 12 out. 2023.

JOHNSON, Don; MENEZES, Alfred; VANSTONE, Scott. \textbf{BThe Elliptic Curve
    Digital Signature Algorithm (ECDSA)}. International Journal of Information
Security, v. 1, n. 1, p. 36-63, 2001. DOI
\url{https://doi.org/10.1007/s102070100002}

KOBLITZ, Ann Hibner, \textit{et. al}. \textbf{Elliptic Curve Cryptography: The
    Serpentine Course of a Paradigm Shift}. \textit{Journal of Number Theory}, vol.
131, no 5, maio de 2011, p. 781–814. DOI.org (Crossref). DOI
\url{https://doi.org/10.1016/j.jnt.2009.01.006}. Disponível em:
\url{https://www.sciencedirect.com/science/article/pii/S0022314X09000481?via%3Dihub}. Acesso em: 21 jan. 2022.

KOBLITZ, Neal. \textbf{Elliptic Curve Cryptosystems}. \textit{Mathematics of
    Computation}, v. 4X, n. 177, p. 20.1-209, 1987. Disponível em:
\url{https://www.ams.org/journals/mcom/1987-48-177/S0025-5718-1987-0866109-5/S0025-5718-1987-0866109-5.pdf}.
Acesso em: 12 out. 2023.

MAIMON, Shir. Elliptic Curve Cryptography. \textbf{University of Rochester},
Rochester, p. 1-13, 10 maio 2018. Disponível em:
https://www.sas.rochester.edu/mth/undergraduate/honorspaperspdfs/shirmaimon2018pdf.
Acesso em: 15 out. 2023.
% https://sas.rochester.edu/mth/undergraduate/honors-papers-plans.html - Página da Rochester

OLIVEIRA, Ronielton Rezende. \textbf{Criptografia simétrica e assimétrica - os
    principais algoritmos de cifragem}. \textit{Segurança Digital} [Revista
    online], v. 31, p. 11-15, 2012. Disponível em:
\url{https://www.ronielton.eti.br/publicacoes/artigorevistasegurancadigital2012.pdf}.
Acesso em: 12 out. 2023.

PABÓN CADAVID, Jhonny Antonio. La criptografía y la protección a la información
digital. \textbf{Revista La Propiedad Inmaterial}, [S. l.], n. 14, p. 59–90,
2010. Disponível em:
\url{https://revistas.uexternado.edu.co/index.php/propin/article/view/2476}.
Acesso em: 12 out. 2023.

REYAD, Omar. Text message encoding based on elliptic Curve Cryptography and a
mapping methodology. \textit{Information Sciences Letters}, v. 7, n. 1, p. 2,
2018. Disponível em:
\url{https://digitalcommons.aaru.edu.jo/cgi/viewcontent.cgi?article=1060&context=isl}.
Acesso em: 12 out. 2023.

SERAGIOTTO, Cesar. \textbf{Criptografia baseada em Identidade}. Disponível em:
\url{http://www.linux.ime.usp.br/~cef/mac499-04/monografias/rec/cesarse/monografia.html}.
Acesso em: 12 out. 2023.

VIANA, Cleberton Junio, \textit{et. al}. Criptografia e segurança.
\textbf{Revista Científica e-Locução}, v. 1, n. 22, p. 30, 19 dez. 2022.
Disponível em:
\url{https://periodicos.faex.edu.br/index.php/e-Locucao/article/view/506}.
Acesso em: 12 out. 2023.

% ---------------------

LENSTRA, H. W. \textbf{Factoring integers with elliptic curves}. \textit{Annals
    of Mathematics}, 1987. Disponível em:
\url{https://wstein.org/edu/Fall2001/124/lenstra/lenstra.pdf}. Acesso em: 12
out. 2023.

% https://www.ripublication.com/ijaer17/ijaerv12n19_140.pdf

