% ==========
% INTRODUÇÃO
% ==========
\section{INTRODUÇÃO}
	
	As curvas elípticas, do seu primeiro uso em criptografia, no trabalho intitulado \textit{Factoring integers with elliptic curves}, do matemático neerlandês Lenstra, H. W, e posterior formulação em corpos de Galois feita de maneira independente por Koblitz, N e Mille, V., até tempos atuais, mostraram-se uma alternativa segura e condizente com a crescente demanda por sintemas com alto grau de segurança. Uma de suas características é prover um elevado nível de criptografia através de chaves privadas relativamente pequenas, o que as tornam atrativas quando considera-se sua utilização em dispositivos embarcados.
	
	Neste trabalho, estudaremos curvas elípticas definidas sobre um corpo finito $\F_p$, com $p > 3$ e primo, de equação $y^2 = x^3 + ax + b$ no qual $a, b \in \F_p$ e $4a^3 + 27b^2 \neq 0$. Curvas que atendem a estas características mostram-se viáveis para uso em criptosistemas por proporcionarem criptografia assimétrica cuja segurança é garantida pelo Problema de Logaritmo Discreto. Investigaremos aqui algumas propriedades destas curvas e aprenderemos as utiliza-las na criptografia e troca de mensagens entre duas pessoas, modelando esta interface de comunicação com a linguagem de programação $Python$.
