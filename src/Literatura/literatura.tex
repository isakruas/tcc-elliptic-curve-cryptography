% =====================
% REVISÃO DA LITERATURA
% =====================
\section{REVISÃO DA LITERATURA}

	\subsection{Equações Diofantinas Lineares} \label{subsec:diofantinas}
	
	Nesta seção, estudaremos equações diofantinas lineares, nosso proposito é apresentar alguns resultados que nos beneficiarão quando formos resolver congruências modulares das curvas elípticas. Parte significativa dos resultados aqui apresentados serão demonstrados, outros, apenas apresentaremos os resultados sem a correspondente demonstração, ficando, a encargo do leitor, a verificação.
	
	\begin{CitacaoLonga}
		Diofanto de Alexandria viveu provavelmente no século III d.C. De sua produção matemática conhecem-se apenas os fragmentos de uma obra que trata de números poligonais e a extremamente original e criativa \textit{Arithmetica}, graças à qual ele é às vezes considerado o pai da álgebra. (DOMINGUES, 2003, p. 49)
	\end{CitacaoLonga}

	\begin{definicao}
		Seja $a_0, a_1, ..., a_n \in \Z$, são chamadas de equações diofantinas lineares, toda equação do tipo $a_0x_0 + a_1x_1 + ... + a_nx_n = d$, no qual, $a_0, a_1, ..., a_n$ são os coeficientes e $x_0, x_1, ..., x_n$ as incógnitas, cujo os valores $x_0, x_1, ..., x_n \in \Z$ 
	\end{definicao}

	Nesta trabalho, nos restringiremos a estudar as equações diofantinas de duas variaveis, isto é, equações $a_0x_0 + a_1x_1 = d$ cujo as soluções $x_0, x_1$ procuradas são números inteiros. O teorema a seguir nos garante que este tipo de equação, quando uma dada condição é satisfeita, admite solução.
	
	\begin{teorema}
	 \textit{(Teorema de Bezout)}: Sejam $a_0, a_1 \in \Z^*$, existe $x_0, x_1 \in \Z$ tal que $a_0x_0 + a_1x_1 = mdc(a_0, a_1)$.
	\end{teorema}

	\begin{proof}
	A demonstração pode ser vista em Ferreira, Domingues (2013, p. 24).
	\end{proof}

	\begin{teorema}
	Seja $a_0x_0 + a_1x_1 = c$ uma equação diofantina, $d = mdc(a_0, a_1)$, a equação dada só terá solução se e somente se $d | c$.
	\end{teorema}

	\begin{teorema}
	\textit{(Algoritmo da divisão de Euclides)}: Seja $a \in \Z$ e $b \in \Z_{+}^{*}$, existe um único par $q, r \in \Z$ tal que $a = qb + r$, donde $0 \leq r < b$.
	\end{teorema}

	\begin{exemplo}
		Seja $a_0 = 5$ e $a_1 = -17$, encontre $x_0$ e $x_1$ da equação $5x_0 - 17x_1 = 9$. 
	\end{exemplo}
